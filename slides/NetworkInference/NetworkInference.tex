\documentclass[10pt, c, xcolor=x11names]{beamer}\usepackage[]{graphicx}\usepackage[]{color}
%% maxwidth is the original width if it is less than linewidth
%% otherwise use linewidth (to make sure the graphics do not exceed the margin)
\makeatletter
\def\maxwidth{ %
  \ifdim\Gin@nat@width>\linewidth
    \linewidth
  \else
    \Gin@nat@width
  \fi
}
\makeatother

\definecolor{fgcolor}{rgb}{0.345, 0.345, 0.345}
\newcommand{\hlnum}[1]{\textcolor[rgb]{0.686,0.059,0.569}{#1}}%
\newcommand{\hlstr}[1]{\textcolor[rgb]{0.192,0.494,0.8}{#1}}%
\newcommand{\hlcom}[1]{\textcolor[rgb]{0.678,0.584,0.686}{\textit{#1}}}%
\newcommand{\hlopt}[1]{\textcolor[rgb]{0,0,0}{#1}}%
\newcommand{\hlstd}[1]{\textcolor[rgb]{0.345,0.345,0.345}{#1}}%
\newcommand{\hlkwa}[1]{\textcolor[rgb]{0.161,0.373,0.58}{\textbf{#1}}}%
\newcommand{\hlkwb}[1]{\textcolor[rgb]{0.69,0.353,0.396}{#1}}%
\newcommand{\hlkwc}[1]{\textcolor[rgb]{0.333,0.667,0.333}{#1}}%
\newcommand{\hlkwd}[1]{\textcolor[rgb]{0.737,0.353,0.396}{\textbf{#1}}}%
\let\hlipl\hlkwb

\usepackage{framed}
\makeatletter
\newenvironment{kframe}{%
 \def\at@end@of@kframe{}%
 \ifinner\ifhmode%
  \def\at@end@of@kframe{\end{minipage}}%
  \begin{minipage}{\columnwidth}%
 \fi\fi%
 \def\FrameCommand##1{\hskip\@totalleftmargin \hskip-\fboxsep
 \colorbox{shadecolor}{##1}\hskip-\fboxsep
     % There is no \\@totalrightmargin, so:
     \hskip-\linewidth \hskip-\@totalleftmargin \hskip\columnwidth}%
 \MakeFramed {\advance\hsize-\width
   \@totalleftmargin\z@ \linewidth\hsize
   \@setminipage}}%
 {\par\unskip\endMakeFramed%
 \at@end@of@kframe}
\makeatother

\definecolor{shadecolor}{rgb}{.97, .97, .97}
\definecolor{messagecolor}{rgb}{0, 0, 0}
\definecolor{warningcolor}{rgb}{1, 0, 1}
\definecolor{errorcolor}{rgb}{1, 0, 0}
\newenvironment{knitrout}{}{} % an empty environment to be redefined in TeX

\usepackage{alltt}

\def\currentCourse{An introduction to graph analysis and modeling}
\def\currentInstitute{Julien Chiquet}
\def\currentLogo{../common_figs/logo_imperial}
\def\currentDate{February, 2019}
\def\currentChapter{Network Inference with Sparse Graphical Models}


% THEME BEAMER
\usepackage{../beamer_theme}

\graphicspath{{figures/},{../common_figs/}}

\usepackage{multirow}
\usepackage{tikz}
\usepackage[vlined]{algorithm2e}

\pgfdeclareimage[width=.5cm]{computer}{computer.png}

\title{\currentCourse}

\subtitle{\huge\currentChapter\normalsize}

\institute{\currentInstitute}

\date{\currentDate}



\AtBeginSection{
  \begin{frame}<beamer>
    \frametitle{Outline}
    \framesubtitle{\insertpart}
    \tableofcontents[currentsection,currentsubsection, subsectionstyle=show/shaded/hide]  
  \end{frame}
}

\AtBeginSubsection{
  \begin{frame}<beamer>
    \frametitle{Outline}
    \framesubtitle{\insertpart}
    \tableofcontents[currentsection,currentsubsection, subsectionstyle=show/shaded/hide]  
  \end{frame}
}

\AtBeginSubsubsection{
  \begin{frame}<beamer>
    \frametitle{Outline}
    \framesubtitle{\insertpart}
    \tableofcontents[currentsection,currentsubsection, subsectionstyle=show/shaded/hide]  
  \end{frame}
}

\newcommand{\dotitlepage}{%
  \begin{frame}
    \titlepage
    \vfill
    \begin{center}
        \scriptsize\url{https://github.com/jchiquet/CourseNetworkLondon}
    \end{center}
    \vfill
    \includegraphics[width=2cm]{\currentLogo}\hfill
    \includegraphics[width=2.5cm]{logo_inra}
  \end{frame}
  %
}

\newcommand{\dotoc}{%
  \begin{frame}
    \frametitle{Outline}
    \tableofcontents[currentsection,
    sectionstyle=show/show,
    subsectionstyle=hide]
  \end{frame}
  %
}

\definecolor{darkred}{rgb}{0.65,0.15,0.25}

\usetikzlibrary{calc,shapes,backgrounds,arrows,automata,shadows,positioning}
\tikzstyle{every state}=[fill=red,draw=none,scale=0.7,font=\small,text=white]
\tikzstyle{every edge}=[-,shorten >=1pt,auto,thin,draw]
\tikzstyle{alertstate}=[fill=mblue]

\pgfdeclareimage[width=.18\textwidth]{microarray}{figures/puce}
\pgfdeclareimage[width=.3\textwidth]{affymetrix}{figures/affy}

\pgfdeclareimage[width=.18\textwidth]{sequencer}{figures/sequencer}
\pgfdeclareimage[width=.25\textwidth]{ngs}{figures/ngs_data}

\pgfdeclareimage[width=.2\textwidth]{rna_seq}{figures/rna_seq}
\pgfdeclareimage[width=.5cm]{computer}{figures/computer}
\IfFileExists{upquote.sty}{\usepackage{upquote}}{}
\begin{document}

\dotitlepage

\definecolor{genecolor}{RGB}{94,135,173}

\begin{frame}
  \frametitle{Statistical analysis of Networks}
  \framesubtitle{Different questions}

  \begin{block}{Understanding the network topology}
    \vspace{-.25cm}
    \begin{itemize}
    \item Data = observed network
    \item Questions: central nodes? cluster structure? small-world property?
    \end{itemize}
  \end{block}
  
  \vfill

  \begin{alertblock}{Inferring/Reconstructing the network}
    \vspace{-.25cm}
    \begin{itemize}
    \item Data = repeated signal observed at each node
    \item Questions: which nodes are connected?
    \end{itemize}
  \end{alertblock}

  \vfill

  \begin{block}{\alert{Each to be combined with}}
    covariates, time, heterogeneous data set, missing data, ...  
  \end{block}
\end{frame}

\begin{frame}
  \frametitle{Reconstruction and analysis of biological networks} 
  \framesubtitle{E. coli regulatory network}  

  \begin{columns}
    \begin{column}{.4\textwidth}
      \begin{small}
        \begin{block}{Target network}
          Relations between genes and their products
          \begin{itemize}
          \item highly structured
          \item always incomplete
          \end{itemize}
        \end{block}
      \end{small}
      \begin{small}
        \begin{block}{Data and method}
          \begin{itemize}
          \item transcriptomic data
          \item \alert{Inference}: sparse Gaussian graphical model
          \item \alert{Analysis}: Stochastic Block Model
          \end{itemize}
        \end{block}
      \end{small}
    \end{column}
    \begin{column}{.55\textwidth}
      \includegraphics[width=\textwidth]{figures/net_reg_ecoli}
    \end{column}
  \end{columns}
\end{frame}

\begin{frame}
  \frametitle{A challenging problem}

  \vspace{-.25cm}

  \begin{columns}[c]
    \begin{column}{.55\textwidth}
      \begin{tikzpicture}[scale=0.75]
        \node[scale=0.75,opacity=0.75] at (-3,3) {\pgfuseimage{sequencer}};
        \node[scale=0.75] at (-3.5,1) {\pgfuseimage{affymetrix}};
        \node[scale=0.75,fill=red, text=white,single arrow] 
        (inference) at (-1.7,1.7) {\sf \scriptsize Inference}; 
        
        \node at (-3,-0.5) {\begin{tabular}{@{}c@{}}
            \tiny $\approx$ 10s/1,000s assays \\ 
            \tiny $\approx $ 1,000s/1,000,000s features \\
          \end{tabular}
        };
        
        %% UN GRAPH 
        \tikzstyle{every edge}=[-,>=stealth',shorten >=1pt,auto,thin,draw,color=genecolor]
        \tikzstyle{every node}=[fill=genecolor]
        \tikzstyle{every state}=[draw=none,text=white,scale=0.5, transform shape] 
        
        % premier cluster
        \node[state] (A1) at (0,1.75) {g1};
        \node[state] (A2) at (1,0.75) {g2};
        \node[state] (A3) at (0,-.25) {g3};
        \node[state] (A4) at (-1,0.75) {g4};
        \node[state] (A5) at (0,0.75) {g5};
        
        \foreach   \name/\angle/\text   in  {B1/234/g6,   B2/162/g7,
          B3/90/g8, B4/18/g9, B5/-54/g10} {
          \node[state,xshift=4cm,yshift=4cm]     (\name)    at
          (\angle:1cm) {\text}; }
        
        \node[state] (B6) at (2,2) {g11};
        \node[state] (C1) at (3,0.5) {g12};
        \node[state] (C2) at (2.2,0) {g13};
        
        \path 
        (A5) edge [bend left] (A1)
        (A5) edge [bend left] (A2)
        (A5) edge [bend left] (A3)
        (A5) edge [bend left] (A4)
        (B6) edge [bend right] (B1) 
        (B6) edge [bend right] (B2) 
        (B6) edge [bend right] (B3) 
        (B6) edge [bend right] (B4) 
        (B6) edge [bend right] (B5) 
        (C2) edge [bend left] (C1)
        (A5) edge [bend left] (B6)
        (B6) edge [bend right] (C2);
      \end{tikzpicture}
    \end{column}
    \begin{column}{.475\textwidth} 
      \begin{block}{Model point of view}
        \vspace{-.25cm}
        \begin{footnotesize}
            \begin{enumerate}
            \item \alert{Nodes} (genes, OTUS, ...)
              \begin{scriptsize}
                \begin{itemize}
                \item \scriptsize fixed variables
                \end{itemize}
              \end{scriptsize}
            \item \alert{Edges} (biological interactions)
              \begin{scriptsize}
                \begin{itemize}
                \item \scriptsize use (partial) correlations or others
                  fancy statistical concepts
                \end{itemize}
              \end{scriptsize}
            \item \alert{Data} (intensities, counts)
              \begin{scriptsize}
                \begin{itemize}
                \item \scriptsize a tidy $n\times p$ dat matrix
                \end{itemize}
              \end{scriptsize}
            \end{enumerate}
            \vspace{-.25cm}
            $\rightsquigarrow$ \alert{Quantities and goals well defined}
          \end{footnotesize}
          \end{block}
    \end{column}
  \end{columns}

  \vspace{-.25cm}
  \begin{block}{Data point of view: \alert{non classical statistics}}<2->
    \vspace{-.25cm}
    \begin{footnotesize}
    \begin{itemize}
    \item (Ultra) High dimensionality ($n<p$, $n\lll p$)
    \item Heterogeneous data
    \end{itemize}
      \end{footnotesize}    
  \end{block}
  \vspace{-.35cm}
  \begin{block}{Biological point of view: \alert{not well defined goals and questions}}<2>
    \vspace{-.25cm}    
    \begin{footnotesize}
    \begin{itemize}
    \item What interaction? Direct? Indirect? Causal?
    \item Whole network? Subnetwork? Groups of key actors?
    \item structured data, mixed data
    \end{itemize}
      \end{footnotesize}    
  \end{block}

\end{frame}

% \begin{frame}
%   \frametitle{Genomic data}
%   \only<1>{\framesubtitle{Microarray technology: parallel measurement of many biological features}}
%   \only<2>{\framesubtitle{Next Generation Sequencing: parallel measurement of \alert{even} many
%     \alert{more} biological features}}
  
%   Focus    e.g.     on    \textit{transcription},    looking    toward
%   \textcolor{genecolor}{\textit{gene regulatory networks}}

%   \begin{tikzpicture}
%     \begin{small}
%       \tikzstyle{every state}=[fill=orange!70!white,draw=none,text=white]
%       \node[state] (dna) at (0,0) {DNA};
%       \node[state] (rna) at (4,0) {RNA};
%       \node[state] (tf) at (6,-0.25) {TF};
%       \node[draw=none,text=white,fill=genecolor, scale=0.75] (gene) at (0.5,0.5) {genes};
%       \node[draw=none,text=genecolor,fill=white] (alter) at ($(dna.south) -(0,3mm)$) {altered?};

      
        
%       \path
%       (dna) edge [->] node[above] {\alert{transcription}} (rna) 
%       (tf) edge [bend left, ->] node[midway] {\textcolor{genecolor}{regulates}} ($(rna.west) -(5mm,0)$)
%       (rna) edge [-,line width=2pt,draw=white,bend left] ($(rna.west) -(15mm,0)$)
%       (rna) edge [bend left, ->] node {\textcolor{genecolor}{regulates}} ($(rna.west) -(15mm,0)$);
%     \end{small}

%     \begin{pgfonlayer}{background}
%       \filldraw [line width=4mm,join=round,black!10]
%       (rna.north -| rna.west) rectangle (rna.south -| rna.east);
%     \end{pgfonlayer}

%   \end{tikzpicture}    

%   \only<1>{
%     \begin{tikzpicture}
%       \node at (0,-.25) {\pgfuseimage{affymetrix}};
%       \node[fill=mred, text=white,single arrow] 
%       (sig) at (3.5,-.25) {\sf \scriptsize signal processing}; 
%       \node[opacity=.75] (array1) at (7.25,0.25) {\pgfuseimage{microarray}};
%       \node[opacity=.9] (array2) at (7.5,0) {\pgfuseimage{microarray}};
%       \node[opacity=.95] (array3) at (7.75,-0.25) {\pgfuseimage{microarray}};
%       \node at (8,-0.5) (array4) {\pgfuseimage{microarray}};
      
%       \begin{pgfonlayer}{background}
%         \filldraw [line width=4mm,join=round,black!10]
%         (array1.north -| array1.west) rectangle (array4.south -| array4.east);
%       \end{pgfonlayer}
      
%       \node at (7,-3) {%
%         $\mathbf{X} = \begin{pmatrix} 
%           x_1^1 & x_1^2 & x_1^3 & \dots & x_1^p \\
%           \vdots \\
%           x_n^1 & x_n^2 & x_1^2 & \dots & x_n^p \\
%         \end{pmatrix}$};
      
%       \node (output) at (0,-3) { 
%         \begin{tabular}{@{}l@{}}
%           \small Matrix of features $n\ll p$\\ \hline
%           \scriptsize Expression levels of $p$ \\
%           \scriptsize probes are simultaneously \\
%           \scriptsize monitored for $n$ individuals
%         \end{tabular}
%       };
      
%       \begin{pgfonlayer}{background}
%         \filldraw [line width=4mm,join=round,black!10]
%         (output.north -| output.west) rectangle (output.south -| output.east);
%       \end{pgfonlayer}
      
%       \node[fill=mred, text=white,single arrow, shape border rotate =180] 
%       (inference) at (3.35,-3) {\sf \scriptsize pretreatment}; 
%     \end{tikzpicture}    
%   }
  
%   \only<2>{
%     \begin{tikzpicture}
%       \node at (0,-.25) {\pgfuseimage{sequencer}};

%       \node[fill=mred, text=white,single arrow] 
%       (sig) at (3.5,-.25) {\sf \scriptsize assembling}; 
      
%       \node[opacity=.75] (array1) at (7.25,0.25) {\pgfuseimage{ngs}};
%       \node[opacity=.9] (array2) at (7.5,0) {\pgfuseimage{ngs}};

%       \begin{pgfonlayer}{background}
%         \filldraw [line width=4mm,join=round,black!10]
%         (array1.north -| array1.west) rectangle (array2.south -| array2.east);
%       \end{pgfonlayer}
      
%       \node at (7,-2.75) {%
%         $\mathbf{X} = \begin{pmatrix} 
%           k_1^1 & k_1^2 & k_1^3 & \dots & k_1^p \\
%           \vdots \\
%          k_n^1 & k_n^2 & k_1^2 & \dots & k_n^p \\
%         \end{pmatrix}$};
    
%       \node (output) at (0,-2.75) { 
%         \begin{tabular}{@{}l@{}}
%           \small Matrix of features $n\lll p$\\ \hline
%           \scriptsize Expression counts are extracted \\
%           \scriptsize from small repeated sequences \\
%           \scriptsize and monitored for $n$ individuals
%         \end{tabular}
%       };

%       \begin{pgfonlayer}{background}
%         \filldraw [line width=4mm,join=round,black!10]
%         (output.north -| output.west) rectangle (output.south -| output.east);
%       \end{pgfonlayer}

%       \node[fill=mred, text=white,single arrow, shape border rotate =180] 
%       (inference) at (3.35,-2.75) {\sf \scriptsize pretreatment}; 
%     \end{tikzpicture}    
%     }
% \end{frame}



\begin{frame}
  \frametitle{Outline}
  \tableofcontents[hideallsubsections]
\end{frame}

\include{ggm}

\section{Network inference with GGM}

\begin{frame}
  \frametitle{Some families of methods for network reconstruction}

  \begin{block}{Test-based methods}
    \begin{itemize}
    \item Tests the nullity of each entries 
    \item Combinatorial problem when $p>30$ \dots
    \end{itemize}    
  \end{block}
  
  \vfill

  \begin{block}{\alert{Sparsity-inducing regularization methods}}
    \begin{itemize}
    \item induce sparsity with the $\ell_1$-norm penalization
    \item Use results from convex optimization
    \item Versatile and computationally efficient
    \end{itemize}
  \end{block}

  \vfill

  \begin{block}{Bayesian methods}
    \begin{itemize}
    \item Compute the posterior probability of each edge
    \item Usually more computationally demanding
    \item For special graphs, computation gets easier
    \end{itemize}
  \end{block}
  
\end{frame}

\pgfdeclareimage[height=0.8\textheight]{sparsity1}{figures/sparsity_1}
\pgfdeclareimage[height=0.8\textheight]{sparsity2}{figures/sparsity_2}
\pgfdeclareimage[height=0.375\textheight]{sparsity4}{figures/sparsity_4}

\begin{frame}
  \frametitle{Inference: maximum likelihood estimator}
  \framesubtitle{The natural approach for parametric statistics}
  
  Let   $X$  be  a   random  vector   with  distribution   defined  by
  $f_{X}(x;\boldsymbol\Theta)$,  where   $\boldsymbol\Theta$  are  the
  model parameters.

  \vfill

  \begin{block}{Maximum likelihood estimator}
    \begin{equation*}
      \hat{\boldsymbol\Theta}      =      \argmax_{\boldsymbol\Theta}
      \ell(\boldsymbol\Theta; \mathbf{X})
    \end{equation*} 
    where  $\ell$ is  the log  likelihood, a  function  of the
    parameters:
    \begin{equation*}
      \ell(\boldsymbol\Theta;      \mathbf{X})      =     \log
      \prod_{i=1}^n f_{X}(\mathbf{x}_i;\boldsymbol\Theta),
    \end{equation*}
    where $\mathbf{x}_i$ is the $i$th row of $\mathbf{X}$.
  \end{block}
  
  \vfill
  
  \begin{block}{Remarks}
    \begin{itemize}
    \item This a convex optimization problem,
    \item We just need to detect non zero coefficients in $\boldsymbol\Theta$
    \end{itemize}
  \end{block}
  
\end{frame}

\begin{frame}
  \frametitle{The multivariate Gaussian log-likelihood }
  
  Let  $\mathbf{S}  =  n^{-1}\mathbf{X}^\intercal \mathbf{X}$  be  the
  empirical variance-covariance  matrix: $\mathbf{S}$ is  a sufficient
  statistic of $ \boldsymbol\Theta$.

  \vfill

  \begin{block}{The log-likelihood}
    \begin{equation*}
      \ell(\boldsymbol\Theta; \mathbf{S}) =
      \frac{n}{2}     \log    \det     (\boldsymbol\Theta)  - \frac{n}{2}
      \mathrm{Trace}(\mathbf{S} \boldsymbol\Theta) + \frac{n}{2}\log(2\pi).
    \end{equation*}
  \end{block}
  
  \vfill
  
  \begin{itemize}
  \item[$\rightsquigarrow$]    The     MLE    $=\mathbf{S}^{-1}$    of
    $\boldsymbol\Theta$ is not defined for $n< p$ and never sparse.
  \item[$\rightsquigarrow$] The need for regularization is huge.
  \end{itemize}
\end{frame}

\begin{frame}
  \frametitle{A Geometric View of Shrinkage}
  \framesubtitle{Constrained Optimization}

  \begin{overlayarea}{\textwidth}{\textheight}
    \begin{columns}
      \begin{column}{0.475\textwidth}
        \begin{tikzpicture}
          \only<1>{%
            \node (Surf) at (0,0) {\pgfuseimage{sparsity1}}
            node     at    (Surf.west)    [rotate=90,yshift=5mm]
            {$L(\theta_1,\theta_2;\mathbf{X})$}
            node at (Surf.south west) [xshift=5mm,yshift=5mm]{$\theta_2$}
            node at (Surf.south east) [xshift=-7.5mm,yshift=2.5mm]{$\theta_1$};
          }
          \only<2>{%
            \node (Surf2) at (0,0) {\pgfuseimage{sparsity2}}
            node    at    (Surf2.west)    [rotate=90,yshift=5mm]
            {$L(\theta_1,\theta_2;\mathbf{X})$}
            node at (Surf2.south west) [xshift=5mm,yshift=5mm]{$\theta_2$}
            node at (Surf2.south east) [xshift=-7.5mm,yshift=2.5mm]{$\theta_1$};
          }
          \only<3->{%
            \node (titi) at (0,0) {\phantom{titi}};
            \node (Surf3) at (0,-4.5) {\pgfuseimage{sparsity4}}
            node at (Surf3.west) [rotate=90,yshift=2.5mm] {$\theta_2$}
            node at (Surf3.south) [yshift=-2.5mm] {$\theta_1$};
          }
        \end{tikzpicture}
      \end{column}
      \begin{column}{0.55\textwidth}
        \only<1>{%
          We basically want to solve a problem of the form
          \begin{equation*}
            \maximize_{\theta_1,\theta_2} \ell(\theta_1,\theta_2;\mathbf{X})
          \end{equation*}
          where $\ell$ is typically a concave likelihood function.
        }
        \only<2->{%
          \begin{equation*}
            \left\{\begin{array}{ll}
                \displaystyle    \maximize_{\theta_1,\theta_2}   &
                \ell(\theta_1,\theta_2;\mathbf{X})\\
                \mathrm{s.t.} & \Omega(\theta_1,\theta_2) \leq c
              \end{array}\right.,
          \end{equation*}
          where  $\Omega$  defines  a  domain  that  \textit{constrains}
          $\boldsymbol\beta$.

          \begin{center}
            How shall we define $\Omega$ ?
          \end{center}
        }
      \end{column}
    \end{columns}
  \end{overlayarea}
\end{frame}

\begin{frame}
  \frametitle{The Lasso}
  \framesubtitle{Least Absolute Shrinkage and Selection Operator}

  \begin{block}{Idea}
    Suggest  an admissible  set that  induces  \alert{sparsity} (force
    several entries to exactly zero in $\hat{\bbeta}$).
  \end{block}

  \vfill

  \begin{overlayarea}{\textwidth}{.4\textheight}
    \begin{columns}
      \begin{column}[c]{.6\textwidth}
        \begin{block}{Lasso as a regularization problem}
          The Lasso estimate of $\bbeta$ is the solution to
          \begin{equation*}
            \hat{\btheta}^{\text{lasso}}     =    \argmin_{\btheta}
            -\ell(\btheta),  \quad   \text{s.t.  }  \sum_{j=1}^p
            \left|\theta_j\right|
            \leq s,
          \end{equation*}
          where $s$ is a shrinkage factor.
        \end{block}
      \end{column}
      \begin{column}{.3\textwidth}
        \includegraphics[width=\textwidth]{figures/lasso_set}
      \end{column}
    \end{columns}
  \end{overlayarea}

\end{frame}

\begin{frame}
  \frametitle{Insights: 2-dimensional example with the square loss}

  \begin{overlayarea}{\textwidth}{\textheight}

    \begin{equation*}
      \sum_{i=1}^n (y_i-x_i^1\theta_1 - x_i^2\theta_2)^2, \qquad
      \only<1>{\text{no constraints}}
      \only<2>{\text{s.t. } |\theta_1| + |\theta_2| < 0.75}
      \only<3>{\text{s.t. } |\theta_1| + |\theta_2| < 0.66}
      \only<4>{\text{s.t. } |\theta_1| + |\theta_2| < 0.4}
      \only<5>{\text{s.t. } |\theta_1| + |\theta_2| < 0.2}
      \only<6>{\text{s.t. } |\theta_1| + |\theta_2| < 0.0743}
    \end{equation*}

  \vspace{-.75cm}

    \includegraphics<1>[width=.9\textwidth]{dess11}
    \includegraphics<2>[width=.9\textwidth]{dess12}
    \includegraphics<3>[width=.9\textwidth]{dess13}
    \includegraphics<4>[width=.9\textwidth]{dess14}
    \includegraphics<5>[width=.9\textwidth]{dess15}
    \includegraphics<6>[width=.9\textwidth]{dess16}

  \end{overlayarea}

\end{frame}

\begin{frame}
  \frametitle{Application to GGM: the "Graphical-Lasso"} 

  \begin{block}{A penalized likelihood approach}
    \vspace{-1em}
    \begin{equation*}
      \hat{\bTheta}_\lambda=\argmax_{\bTheta \in \mathbb{S}_+}
      \ell(\bTheta;\mathbf{X})-\lambda
      \|\bTheta\|_{\ell_1}
    \end{equation*}
  where
  \begin{itemize}
  \item $\mathcal{\ell}$ is the model log-likelihood,
  \item $\|\cdot\|_{\ell_1}$ is a \alert{penalty function} tuned by
    $\lambda>0$. 
    \vfill
      \begin{enumerate}
      \item \textit{regularization} (needed when $n \ll p$), 
      \item \textit{selection} (sparsity induced by the $\ell_1$-norm),
      \end{enumerate}
    \item     solved    in     \texttt{R}-packages    \textbf{glasso},
      \textbf{quic}, \textbf{huge} ($\mathcal{O}(p^3)$)
  \end{itemize}
\end{block}

\end{frame}

\begin{frame}
  \frametitle{Application to GGM: "Neighborhood selection"} 

  A close cousin, thank to the relationship between Gaussian vector and linear regression
  
  \textcolor{gray}{
  Remember that
  \begin{equation*}
    X_i | X_{ \setminus i} = \sum_{\alert{j \in \text{neighbors}(i)}} \beta_j X_j + \varepsilon_i
    \quad         \text{with         }         \beta_j         =
    -\frac{\Theta_{ij}}{\Theta_{ii}}.
  \end{equation*}
  }
  
  \begin{block}{A penalized least-square approach}
    Let $\bX_i$ be the $i$th column of the data matrix (i.e data associated to variable (gene) $i$), and $\bX_{\backslash i}$ deprived of colmun $i$. We select the neighbors of variable $i$ by solving
    \begin{equation*}
      \widehat{\boldsymbol\beta}^{(i)} = \argmin_{\boldsymbol\beta \in \mathbb{R}^{p-1} }
      \frac{1}{n} \left\| \mathbf{X}_i - \mathbf{X}_{\backslash i} \,
        {\boldsymbol\beta} \right\|_2^2 + \lambda \left\| {\boldsymbol\beta} \right\|_{1}
    \end{equation*}

    \begin{itemize}
    \item[\textcolor{red}{$-$}] not symmetric, not positive-definite
    \item[\textcolor{green}{$+$}] $p$
      Lasso solved with Lars-like algorithms ($\mathcal{O}(npd)$ for $d$ neighbors).
    \end{itemize}

\end{block}

\end{frame}


% \begin{frame}
%   \frametitle{Network inference for count data}
%   \framesubtitle{Data transformation}
% 
%   Consider $\bX = (X^1,\dots,X^n)$ some count data with size $n\times p$. 
% 
%   \begin{block}{Simple transformation}
%     Often surprisingly efficients
%     \begin{itemize}
%     \item log transformation $\log(1 + \bX)$
%     \item compute $\boldsymbol S_n$ by means of \alert{Spearman's correlation}
%     \end{itemize}
%   \end{block}
% 
%   \vfill
% 
%   \begin{block}{Non paranormal transformation (Liu et al 2009)}
%     The random vector $X$ has non-paranormal distribution if there exist \[f(X)=f(X_1,\dots, X_p)
%     \sim\mathcal{N}(\boldsymbol\mu,\boldsymbol\Theta^{-1}).\]
%     \begin{itemize}
%     \item Distribution of $X$  is a  \alert{Gaussian copula} if  $f$ is
%       monotone differentiable
%     \item    $X_i    \indep    X_j   |    X_{\backslash    i,j}$    iff
%       $\boldsymbol\Theta_{ij} = 0$.
%     \end{itemize}
%   \end{block}
% 
% \end{frame}

% \begin{frame}
%   \frametitle{Network inference for count data}
%   \framesubtitle{Poisson graphical models} 
%   
%   \begin{block}{Poisson graphical Lasso (Allen et al, 2012)}
%     Assuming  that   $X_j  |   X_k  \sim   \mathcal{P}(\exp(\beta_j  +
%     \sum_{j\neq k} \beta_k X_k ))$
%     \begin{equation*}
%       \hat{\bbeta}            \argmin_{\hat{\bbeta}\in\Rset^p}           
%       \left\{ -\sum_{i=1}^n\sum_{k\neq j} X_{ij} X_{ik}\beta_k -
%         \exp\{X_{ik}\beta_k \} \right\} + \lambda \|\bbeta \|_1.
%     \end{equation*}
%     \begin{itemize}
%     \item[$\rightsquigarrow$]    \alert{Log-linear    version}    of
%       neighborhood selection
%     \item[$\rightsquigarrow$]  Other extensions  in Yang  et al,  2014
%       (truncated Poisson).
%     \end{itemize}
%   \end{block}
% 
%   \vfill
%   
%   \begin{itemize}
%   \item[\textcolor{green}{$+$}] Better performance than GGM\dots
%   \item[\textcolor{red}{$-$}] \dots on simulated Poisson data
%   \item[\textcolor{red}{$-$}] Computationally less efficient
%   \end{itemize}
%   
% \end{frame}
% 
% \begin{frame}
%   \frametitle{Dealing with the growing number of feature}
%   
%   \begin{block}{Problem}
%     The number of OTU $p$ may be huge in metagenomics studies 
%     \begin{itemize}
%     \item Statistical limitation (depends on $d,n$)
%     \item Computational limitation (depends on your time but max. 1e6)
%     \end{itemize}
%   \end{block}
% 
%   \vfill
%   
%   \begin{block}{How should we limit the size of the problem?}
%     \begin{itemize}
%     \item Screening (discarding of irrelevant variables)
%     \item Clustering (aggregation of similar actors)
%     \end{itemize}
%     $\rightsquigarrow$ How does this affect the inferred networks?
%   \end{block}
%     
% \end{frame}


\include{limitations}

\begin{frame}
  \frametitle{Extensions motivated by biological data}

  \begin{block}{\alert{Strengthen the inference } by}
    \vspace{-.25cm}

    \begin{itemize}
    \item accounting for biological features

      \begin{enumerate}
      \item \alert{structure} of the network (organization of biological mechanisms)
      \item sample \alert{heterogeneity} (structure of the population)
      \item horizontal \alert{integration} (use multiple data and platforms)
      \item Deal with \alert{covariates}
      \end{enumerate} 

    \item accounting for data features 

      \begin{enumerate}
      \item What if some \alert{important actor is missing}?
      \item Extend to \alert{non strictly normal} distribution
      \item Deal with a \alert{large number} of actors
      \end{enumerate}
    \end{itemize}

  $\rightsquigarrow$ How? Essentially by crafting the regularization according to our prior knowledge  

  \end{block}

\end{frame}

%% account for underlying structure of the network
\include{simone}

%% account for sample heterogeneity
\include{multitask}

%% account for multivariate data
\section{Accouting for multiscale data with multiattribute models}

\begin{frame}
  \frametitle{Why Multi-attribute Networks?}
  \framesubtitle{Joint work with E. Kolaczyk (Boston) and C.  Ambroise
    (Évry)}

  \begin{tikzpicture}
    \tikzstyle{every state}=[fill=orange!70!white,draw=none,text=white]
      
    \node[state] (dna) at (0,0) {DNA};
    \node[state] (rna) at (4,0) {RNA};
    \node[state] (proteins) at (8,0) {Proteins};
    \node[state] (tf) at (6,-1.2) {TF};
    \node[state] (enzyme) at (9,-2) {Enz.};
    \node[draw=none,text=white,fill=genecolor, scale=0.75] (gene) at (0.5,0.5) {genes};
    
    \path
    (dna) edge [->] node[above] {transcription} (rna) 
    (rna) edge [->] node[above] {translation} (proteins) 
    (dna) edge [loop left,->] node[below=10pt] {replication} (dna) 
    (proteins) edge [->] node {} (tf) 
    (proteins) edge [->] node {} (enzyme) 
    (proteins) edge [loop right, ->] node[above left=10pt] {\textcolor{red}{may bind}} (proteins) 
    (tf) edge [bend left, ->] node[midway] {\textcolor{genecolor}{regulates}} ($(rna.west) -(5mm,0)$)
    (rna) edge [-,line width=2pt,draw=white,bend left] ($(rna.west) -(15mm,0)$)
    (rna) edge [bend left, ->] node {\textcolor{genecolor}{regulates}} ($(rna.west) -(15mm,0)$);
  \end{tikzpicture}    

  \vspace{-.5cm}

  \begin{block}{Data integration}
    \begin{itemize}
    \item  Omic technologies  can  profile  cells at  \alert{different
        levels}: DNA, RNA, protein, chromosomal, and functional.
    \item \alert{multiple} molecular  profiles \alert{combined} on the
      same set of biological samples can be \textit{synergistic}.
    \end{itemize}
  \end{block}
  
\end{frame}

\begin{frame}
  \frametitle{Multiattribute GGM}

    Consider e.g. some $p$ genes of interest and the $K=2$ omic experiments
    \begin{enumerate}
    \item $X_{i1}$ is the expression profile of gene $i$ (transcriptomic data),
    \item $X_{i2}$ is the corresponding protein concentration (proteomic data).
    \end{enumerate}

    \vfill

  \begin{block}{Define a block-wise precision matrix}
      \vspace{-.25cm}
      \begin{itemize}
      \item  $X =  (X_1,  \dots,  X_p)^T \sim  \mathcal{N}(\mathbf{0},
        \bSigma)$ in $\Rset^{pK}$,
      \item $X_i=(X_{i1},\dots,X_{iK})^\intercal \in \mathbb{R}^K$.
      \end{itemize}
      \[
      \invcov = \bSigma^{-1} = \begin{bmatrix}
        \invcov_{11} & & \invcov_{1p} \\
        & \ddots & \\
        \invcov_{p1} & & \invcov_{pp} \\
      \end{bmatrix}, \qquad  \invcov_{ij} \in \mathcal{M}_{K,K},
      \ \forall (i,j)\in\mathcal{P}^2.
      \]
    \end{block}

    \vfill

    \begin{beamerboxesrounded}[upper=sur:head,lower=sur:bloc,shadow=true]{Graphical Interpretation}
      Define  $\mathcal{G}=(\mathcal{P},\mathcal{E})$   as  \alert{the
        multivariate analogue} of the {\it conditional graph}:
      \vspace{-.25cm}
      \begin{equation*}
        (i,j)\in    \mathcal{E}    \Leftrightarrow    \invcov_{ij}    \ne
        \mathbf{0}_{KK}.
      \end{equation*}
    \end{beamerboxesrounded}

\end{frame}

\begin{frame}
  \frametitle{Multiattribute GGM as multivariate regression}

  \begin{block}{Multivariate analysis view point}
    Straightforward algebra and we have
    \begin{equation*}
      \label{eq:condK}
      X_j \, |\,  X_{\backslash j}  = x \sim  \mathcal{N}(- \invcov_{jj}^{-1}\invcov_{j
        \backslash j} x , \invcov_{ii}^{-1})\enskip.
    \end{equation*}
    or   equivalently,    letting   $\displaystyle    \mathbf{B}_j^T   =
    -\invcov_{jj}^{-1} \invcov_{i\backslash j}$,
    \begin{equation*}
      \label{eq:condK}
      X_j \, |\, X_{\backslash j} = \mathbf{B}_j^T X_{\backslash j} +
      \boldsymbol\varepsilon_j \quad \boldsymbol\varepsilon_j
      \sim \mathcal{N}(\mathbf{0},\invcov_{ii}^{-1}), \quad \boldsymbol\varepsilon_j \perp X.
    \end{equation*}
  \end{block}

    \begin{colormixin}{60!white}
      \begin{block}{Remembering the univariate case?}
        \[
        X_j   |    X_{   \setminus    j}   =   -    \sum_{\alert{k   \in
            \text{neighbors}(j)}} \frac{\invcov_{jk}}{\invcov_{jj}}  X_j +
        \varepsilon_j,\quad              \varepsilon_j              \sim
        \mathcal{N}(0,\invcov_{jj}^{-1}), \quad \varepsilon_j \perp X.
        \]
      \end{block}
    \end{colormixin}
\end{frame}

\begin{frame}
  \frametitle{Multivariate neighborhood selection} 

  \begin{block}{The penalized multivariate regression approach}
    For each node /gene, recover its neighborhood by solving 
    \begin{equation*}
      \arg  \min_{\mathbf{B}_j  \in  \mathcal{M}_{(p-1)K,K}}  \frac{1}{2n} \left\|
        \bX_j - \bX_{\backslash j}\bB_j\right\|_F^2 +
      \lambda \ \text{Pen}(\bB_j),
    \end{equation*}
  \end{block}
  
  \vfill
  
  \begin{block}{Choice of penalty}
    Group-based   penalty   to   activate  the   set   of   attributes
    simultaneously on a given link:
    \begin{equation*}
      \text{Pen}(\bB_j) =       \sum_{k \neq j}  \|\bB_j^{(k)}\|  \enskip  ,
      \quad \bB_j^{(k)} \in \mathcal{M}_{KK}
    \end{equation*}
    \begin{itemize}
    \item  \alert{$\|M\|=   \|M\|_F=\left(  \sum_{i,j}  M_{ij}^2\right)^{1/2}$,
        the Frobenius norm},
    \item  $\|M\|=  \|M\|_\infty=  \max_{i,j}{|M_{ij}|}$, the  sup  norm
      (shared magnitude),
    \item $\|M\|= \|M\|_\star=\sum \mathrm{eig}(M)$, the nuclear norm
      (rank penalty).
    \end{itemize}      
  \end{block}
\end{frame}

\begin{frame}
  \frametitle{Breast cancer data: application}

  Two cohorts with both proteomic and transcriptomic data
  \begin{enumerate}
    \item \emphase{NCI-60}: $n=60$ diverse human cancer cell lines, $p=91$
    \item  \emphase{RATHER}: $n=100$ sample from patients with breast cancer, $p=117$
  \end{enumerate}

\begin{figure}[htbp!]
  \centering
  \begin{tabular}{@{}cc@{}}
   RATHER & NCI-60 \\
    \includegraphics[width=.35\textwidth]{figures/jaccard_RATHER}
  & \includegraphics[width=.5\textwidth]{figures/jaccard_NCI60}
  \end{tabular}
  \caption{Jaccard's similarity index
    $J(A,B) = \frac{\left|A\cap B\right|}{\left|A\cup B\right|}$
    between uni-attribute and multiattribute networks, for RATHER and
    NCI60 data set: multiattribute networks share a high Jaccard
    index with both uni-attribute networks.}
  \label{fig:jaccard}
\end{figure}

\end{frame}

\begin{frame}
  \frametitle{Inferred networks}
  
\begin{figure}[htbp!]
  \centering
  \begin{tabular}{@{}lccc@{}}
    & proteomic network  & transcriptomic network  & multiattribute network \\
    \rotatebox{90}{\hspace{1.2cm}NCI60} 
    & \includegraphics[width=.25\textwidth]{figures/protNet_NCI60}
    & \includegraphics[width=.25\textwidth]{figures/exprNet_NCI60}
    & \includegraphics[width=.25\textwidth]{figures/bivarNet_NCI60} \\
    \rotatebox{90}{\hspace{1.2cm}RATHER} 
    & \includegraphics[width=.25\textwidth]{figures/protNet_RATHER}
    & \includegraphics[width=.25\textwidth]{figures/exprNet_RATHER}
    & \includegraphics[width=.25\textwidth]{figures/bivarNet_RATHER} \\
  \end{tabular}
  \caption{Uni-attribute and multiattribute networks inferred on both
    NCI60 and RATHER dataset. The number of neighbors of each entity
    is chosen by cross-validation. Multiattribute networks catch motif
    found in the uniattribute counterparts.}
  \label{fig:networks}
\end{figure}

\end{frame}



%% Handling count data-> Poisson Log-normal

\section{Model for count data}

\begin{frame}[fragile]
  \frametitle{Motivations: oak powdery mildew pathobiome}

  \begin{block}{Metabarcoding data from [JFS16]}<1->
    \begin{itemize}
    \item $n = 116$ leaves, $p = 114$ species ($66$ bacteria, $47$ fungies + \textit{E. alphitoides})
\begin{knitrout}\scriptsize
\definecolor{shadecolor}{rgb}{0.969, 0.969, 0.969}\color{fgcolor}\begin{kframe}
\begin{verbatim}
##       f_1 f_2 f_3 f_4 E_alphitoides b_1045 b_109 b_1093
## A1.02  72   5 131   0             0      0     0      0
## A1.03 516  14 362   0             0      0     0      0
## A1.04 305  24 238   0             0      0     0      0
\end{verbatim}
\end{kframe}
\end{knitrout}
    \item $d = 8$ covariates (tree susceptibility, distance to trunk, orientation, \dots)
\begin{knitrout}\scriptsize
\definecolor{shadecolor}{rgb}{0.969, 0.969, 0.969}\color{fgcolor}\begin{kframe}
\begin{verbatim}
##         treeStatus orientation branch distToTrunk
## A1.02 intermediate          SW      1         202
## A1.03 intermediate          SW      1         175
## A1.04 intermediate          SW      1         168
\end{verbatim}
\end{kframe}
\end{knitrout}
    \item Sampling effort in each sample (bacteria $\neq$  fungi)
\begin{knitrout}\scriptsize
\definecolor{shadecolor}{rgb}{0.969, 0.969, 0.969}\color{fgcolor}\begin{kframe}
\begin{verbatim}
##      [,1] [,2] [,3] [,4] [,5] [,6] [,7] [,8]
## [1,] 2488 2488 2488 2488 2488 8315 8315 8315
## [2,] 2054 2054 2054 2054 2054  662  662  662
## [3,] 2122 2122 2122 2122 2122  480  480  480
\end{verbatim}
\end{kframe}
\end{knitrout}
    \end{itemize}
  \end{block}

\end{frame}

\begin{frame}
  \frametitle{Problematic \& Basic formalism}
  
  \begin{block}{Data tables: $\bY = (Y_{ij}), n \times p$;  $\bX = (X_{ik}), n \times d$; $\bO = (O_{ij}), n \times p$ where}
    \vspace{-.25cm}
    \begin{itemize}
    \item $Y_{ij} = $ abundance (read counts) of species (genes) $j$ in sample $i$
    \item $X_{ik} = $ value of covariate $k$ in sample $i$
    \item $O_{ij} = $ offset (sampling effort) for species $j$ in sample $i$
    \end{itemize}
  \end{block}

  \vfill

  \begin{block}{Need for multivariate analysis to}
    \vspace{-.25cm}
    \begin{itemize}
    \item understand \emphase{between-species/genes interactions} \\
      \rsa 'network' inference (variable/covariance selection)
    \item correct for technical and \emphase{confounding effects} \\
      \rsa account for covariables and sampling effort
    \end{itemize}
  \end{block}

  \rsa need a generic framework to \alert{model dependences between count variables}

\end{frame}

%====================================================================

\begin{frame}{Models for multivariate count data}
\begin{small}

  \begin{block}{If we were in a Gaussian world, the \alert{general linear model} would be appropriate}<1->
    For each sample $i = 1,\dots,n$, it explains 
    \begin{itemize}
    \item the abundances of the $p$ species ($\bY_i$) 
    \item by the values of the $d$ covariates $\bX_i$ and the $p$ offsets $\bO_i$
    \end{itemize}
    \begin{equation*}
      \bY_i = 
      \underbrace{\bX_i \mathbf{B}}_{\begin{tabular}{c} \text{account for} \\ \text{covariates}  \end{tabular}} 
      + \underbrace{\bO_i}_{\begin{tabular}{c} \text{account for} \\ \text{sampling effort}  \end{tabular}}
      \ + \bvarepsilon_i, 
      \ \bvarepsilon_i \sim \mathcal{N}(\bzr_p, \underbrace{\bSigma}_{\begin{tabular}{c} \text{\emphase{dependence}} \\ \text{\emphase{between species}}  \end{tabular}})
    \end{equation*}
    \begin{itemize}
      \item[\textcolor{mred}{+}] \only<1>{\emphase{null covariance $\Leftrightarrow$ independence \rsa uncorrelated species do not interact}}
      \only<2>{\sout{\emphase{null covariance $\Leftrightarrow$ independence \rsa uncorrelated species do not interact}}}
    \end{itemize}
  \end{block}

  \begin{block}{But we are not, and there is no generic model for multivariate counts}<2>
    \begin{itemize}
      \item Data transformation ($\log{}, \sqrt{} $) : quick and dirty \\
      \item Non-Gaussian multivariate distributions: do not scale to data dimension yet \\
      \item \emphase{Latent variable models}: interaction occur in a latent (unobserved) layer\\
    \end{itemize}
  \end{block}

\end{small}
    
\end{frame}

%====================================================================
\begin{frame}{{P}oisson-log normal (PLN) distribution}

\begin{small}

  \begin{block}{A latent Gaussian model}<1>
  Originally proposed by Atchisson [AiH89]
  \[
    \bZ_i \sim \mathcal{N}(\bzr, \boldsymbol\Sigma)
  \]
  \[
    \bY_i \,|\, \bZ_i \sim \clP(\exp{\{\alert{\bO_i + \bX_i^\intercal \bB} + \bZ_i\}})
  \]
  \end{block}

  \vfill
  \vspace{-.25cm}

  \begin{block}{Interpretation}
    \vspace{-.25cm}
  \begin{itemize}
   \item Dependency structure encoded in the latent space (i.e. in $\bSigma$)
   \item Additional effects are fixed
   \item Conditional Poisson distribution = noise model
  \end{itemize}
  \end{block}

  \vspace{-.25cm}

  \begin{block}{Properties}
    \vspace{-.25cm}
      \begin{itemize}
        \item[\textcolor{green}{+}] over-dispersion
        \item[\textcolor{green}{+}] covariance with arbitrary signs
        \item[\textcolor{mred}{-}] maximum likelihood via EM algorithm is limited to a couple of variables
      \end{itemize}
  \end{block}

\end{small}

\end{frame}

%%%%%%%%%%%%%%%%%%%%%%%%%%%%%%%%%%%%%%%%%%%%%%%%%%%%%%%%%
\begin{frame}{Geometrical view}

\begin{knitrout}\scriptsize
\definecolor{shadecolor}{rgb}{0.969, 0.969, 0.969}\color{fgcolor}
\includegraphics[width=.8\textwidth]{figures/PLN_geom-1} 

\end{knitrout}

\end{frame}

\begin{frame}
  \frametitle{Our contributions}

  \begin{block}{Algorithm/Numerical}
    A variational approach coupled with convex optimization techniques suited to higher dimensional data sets.
    
    \paragraph{{\tt PLNmodels} R/C++-package:}  \text{\url{https://github.com/jchiquet/PLNmodels}}
  \end{block}
  
  \vfill
  
  \begin{block}{Extensions for multivariate analysis}
     \paragraph{Idea:} put some additional constraint on the residual variance.
      \begin{itemize}
      \item \emphase{Network Inference} \\
        \rsa select direct interaction in $\bSigma^{-1}$ via sparsity constraints
        
      \item \textcolor{gray}{\it Principal component analysis}\\
        \textcolor{gray}{\it constraint the rank of $\bSigma$ (most important effect in the variance)}

    \end{itemize}

    \paragraph{Challenge:} a variant of the variational algorithm is required for each model
  \end{block}


\end{frame}



\begin{frame}[fragile]
  \frametitle{PLN-network: unravel important interactions}

  \paragraph{Variable selection of direct effects.}
    \begin{align*}
      \bZ_i \text{ iid} & \sim \clN_p(\bzr_p, \bSigma), & \emphase{\|\bSigma^{-1}\|_1 \leq c} \\
      \bY_i \,|\, \bZ_i & \sim \clP(\exp\{\bO_i + \bX_i \bbeta + \bZ_i\})
      \end{align*}

  \paragraph{Interpretation: conditional independence structure.}
    \begin{equation*}
      (i,j)  \notin  \mathcal{E}  \Leftrightarrow  Z_i  \indep  Z_j  |
      Z_{\backslash \{i,j\}} \Leftrightarrow \bSigma_{ij}^{-1} = 0.
    \end{equation*}

    \begin{center}
      \begin{tabular}{c@{\hspace{2cm}}c}
        \begin{tabular}{c}
          \small $\mathcal{G}=(\mathcal{P},\mathcal{E})$ \\
          \includegraphics[width=.3\textwidth]{graph}
        \end{tabular}
     &
       \begin{tabular}{c}
         \small $\bSigma^{-1}$\\\includegraphics[width=.2\textwidth]{Markovadjacency}
       \end{tabular}
      \end{tabular}
    \end{center}

  \paragraph{PLN-network: find a sparse reconstruction of the latent inverse covariance}
  
    Iterate over variational estimator and Graphical-Lasso [BDE08,YL08,FHT07] in the latent layer

\end{frame}

\begin{frame}[fragile]
  \frametitle{Networks of partial correlations for oak mildew pathobiome}

\begin{knitrout}\scriptsize
\definecolor{shadecolor}{rgb}{0.969, 0.969, 0.969}\color{fgcolor}\begin{kframe}
\begin{alltt}
  \hlcom{# Models with offset and covariates (tree + orientation)}
  \hlstd{formula} \hlkwb{<-} \hlstd{counts} \hlopt{~} \hlnum{1} \hlopt{+} \hlstd{covariates}\hlopt{$}\hlstd{tree} \hlopt{+} \hlstd{covariates}\hlopt{$}\hlstd{orientation} \hlopt{+} \hlkwd{offset}\hlstd{(}\hlkwd{log}\hlstd{(offsets))}
  \hlstd{models} \hlkwb{<-} \hlkwd{PLNnetwork}\hlstd{(formula,} \hlkwc{penalties} \hlstd{=} \hlnum{10}\hlopt{^}\hlkwd{seq}\hlstd{(}\hlkwd{log10}\hlstd{(}\hlnum{2}\hlstd{),} \hlkwd{log10}\hlstd{(}\hlnum{0.6}\hlstd{),} \hlkwc{len} \hlstd{=} \hlnum{30}\hlstd{))}
\end{alltt}
\end{kframe}
\end{knitrout}

\begin{overprint}
  \foreach\x in{1,...,15}{
    \includegraphics<\x>[trim={2.5cm 2.5cm 2.5cm 2.5cm},clip, width=\textwidth]{figures/plot-networks-\x}
  }
\end{overprint}

\end{frame}

\end{document}
